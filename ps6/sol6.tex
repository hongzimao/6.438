\documentclass{article}

\usepackage[english]{babel}
\usepackage[utf8]{inputenc}
\usepackage{amsmath}
\usepackage{amsthm}
\usepackage{amssymb}
\usepackage{mathtools}
\usepackage{amsfonts}
\usepackage{subcaption}
\usepackage{graphicx}
\usepackage{wrapfig}
\usepackage{bbm}
\usepackage{dsfont}
\usepackage{listings}

% set up margin
\usepackage
[
  a4paper,
  left=3cm,
  right=3cm,
  top=3cm,
  bottom=3cm,
]
{geometry}

% set up header
\usepackage{fancyhdr}
\pagestyle{fancy}
\fancyhf{}
\lhead{6.438 Algorithms for Inference}
\chead{Problem Set 6}
\rhead{Hongzi Mao}
\cfoot{\thepage}
\rfoot{\footnotesize{\emph{Collaborated with: Hongzhou Ye, Zhiwei Ding}}}

% footer line
\renewcommand{\footrulewidth}{0.4pt}

% sans serif italic
\newcommand{\s}[1]{\textsf{\textit{#1}}}

% bold face sans serif
\newcommand{\bs}[1]{\textsf{\textbf{#1}}}

% set symbol
\usepackage[mathscr]{euscript}

% empty set
\let\emptyset\varnothing

% qed
\newcommand{\qeds}{\hfill\qedsymbol}

% math bold face
\newcommand{\bm}{\mathbf}

% argmax
\DeclareMathOperator*{\argmax}{argmax}
\DeclareMathOperator*{\argmin}{argmin}

% colorful reference
\usepackage{hyperref}
\usepackage{color}
\definecolor{darkred}{rgb}{0.7,0,0}
\definecolor{darkgreen}{rgb}{0,0.5,0}
\hypersetup{colorlinks=true,
        linkcolor=darkred,
        citecolor=darkgreen}
\urlstyle{same}

% independence symbol
\makeatletter
\newcommand*{\indep}{%
  \mathbin{%
    \mathpalette{\@indep}{}%
  }%
}
\newcommand*{\nindep}{%
  \mathbin{%                   % The final symbol is a binary math operator
    \mathpalette{\@indep}{\not}% \mathpalette helps for the adaptation
                               % of the symbol to the different math styles.
  }%
}
\newcommand*{\@indep}[2]{%
  \sbox0{$#1\perp\m@th$}%        box 0 contains \perp symbol
  \sbox2{$#1=$}%                 box 2 for the height of =
  \sbox4{$#1\vcenter{}$}%        box 4 for the height of the math axis
  \rlap{\copy0}%                 first \perp
  \dimen@=\dimexpr\ht2-\ht4-.2pt\relax
  \kern\dimen@
  {#2}
  \kern\dimen@
  \copy0 %                       second \perp
} 
\makeatother

%%%%%%%%%%%%%%%%%%%%%%%%%%%%%%%%%%%%%%%%%%%%%%%%%%%%%%%%%%%%%%%%%%%%%%%%
%%%%%%%%%%%%%%%%%%%%%%%%% Begin document here %%%%%%%%%%%%%%%%%%%%%%%%%%
%%%%%%%%%%%%%%%%%%%%%%%%%%%%%%%%%%%%%%%%%%%%%%%%%%%%%%%%%%%%%%%%%%%%%%%%
\begin{document}

\section*{Problem 6.1}
%
(a) $\s{x}[n]$ is a Markov process. This is because $\s{x}[n]$ at each step
$n$ only depends on the random variable $\s{a}, \omega, \phi$ within
this step locally. This trivially leads to the Markov property that $\s{x}[n]$
is independent of $\s{x}[0], \s{x}[1], \cdots, \s{x}[n-2]$ condition on $\s{x}[n-1]$.
\\

\noindent
(b) $\s{x}_1[2n]$ is a Markov process. To show Markov property, we need to show
$p(x_1[2n] \,\big|\, x_1[2n - 2]) = p(x_1[2n] \,\big|\, x_1[2n - 2], x_1[2n -4], \cdots)$. We note that
\begin{align*}
p(x_1[2n] \,\big|\, x_1[2n - 2])	 &=
\sum_{x_1[2n - 1]} p(x_1[2n], x_1[2n-1] \,\big|\, x_1[2n - 2])\\
&= \sum_{x_1[2n - 1]} p(x_1[2n] \,\big|\, x_1[2n-1], x_1[2n-2])
p(x_1[2n-1] \,\big|\, x_1[2n-2])\\
&= \sum_{x_1[2n - 1]} p(x_1[2n] \,\big|\, x_1[2n-1], x_1[2n-2], \cdots)
p(x_1[2n-1] \,\big|\, x_1[2n-2], \cdots)\\
&\;\;\;\;\text{(since $\s{x}_1[0], \s{x}_1[1], \cdots, \s{x}_1[2n-1], \s{x}_1[2n]$ is a Markov process)}\\
&= \sum_{x_1[2n - 1]} p(x_1[2n] \,\big|\, x_1[2n-1], x_1[2n-2], x_1[2n-4], x_1[2n-6], \cdots)\\
&\;\;\;\; \times p(x_1[2n-1] \,\big|\, x_1[2n-2], x_1[2n-4], x_1[2n-6], \cdots)\\
&= \sum_{x_1[2n - 1]} p(x_1[2n], x_1[2n-1] \,\big|\, x_1[2n-2], x_1[2n-4], x_1[2n-6], \cdots)\\
& = p(x_1[2n] \,\big|\, x_1[2n-2], x_1[2n-4], x_1[2n-6], \cdots)
\end{align*} \qeds

$\s{y}[n]$ is not a Markov process. This is because $\s{y}[2n]$ is a function of
$\s{y}[2n-2]$ by construction from $\s{x}_1$. Since $\s{y}[2n]$ is independent of
$\s{y}[2n-1]$, $\s{y}[2n]$ would not be conditional independent of $\s{y}[2n-2]$ on
$\s{y}[2n-1]$. As a concrete example, let the initial state
$\s{x}[0] \sim \text{Bern}(1/2)$,
$\s{y}[0] \sim \text{Bern}(1/2)$ and the Markov process be
$\s{x}[0] = \s{x}[1] = \cdots = \s{x}[n]$, 
$\s{y}[0] = \s{y}[1] = \cdots = \s{y}[n]$. Then
\begin{align*}
	p(y[2n] \,\big|\, y[2n - 1]) = \frac{1}{2}
	\neq \mathds{1}_{y[2n] = y[2n - 2]} = p(y[2n] \,\big|\, y[2n - 1], y[2n - 2]).
\end{align*} \qeds
\pagebreak

%%%%%%%%%%%%%%%%%%%%%%%%%%%%%%%%%%%%%%%%%%%%%%%%%%%%%%%%%%%%%%%%%%%%%%%% 
\section*{Problem 6.2}
















\end{document}