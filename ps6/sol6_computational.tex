\documentclass{article}

\usepackage[english]{babel}
\usepackage[utf8]{inputenc}
\usepackage{amsmath}
\usepackage{amsthm}
\usepackage{amssymb}
\usepackage{mathtools}
\usepackage{amsfonts}
\usepackage{subcaption}
\usepackage{graphicx}
\usepackage{wrapfig}
\usepackage{bbm}
\usepackage{dsfont}
\usepackage{listings}

% set up margin
\usepackage
[
  a4paper,
  left=3cm,
  right=3cm,
  top=3cm,
  bottom=3cm,
]
{geometry}

% set up header
\usepackage{fancyhdr}
\pagestyle{fancy}
\fancyhf{}
\lhead{6.438 Algorithms for Inference}
\chead{Problem Set 6}
\rhead{Hongzi Mao}
\cfoot{\thepage}
\rfoot{\footnotesize{\emph{Collaborated with: Hongzhou Ye, Zhiwei Ding}}}

% footer line
\renewcommand{\footrulewidth}{0.4pt}

% sans serif italic
\newcommand{\s}[1]{\textsf{\textit{#1}}}

% bold face sans serif
\newcommand{\bs}[1]{\textsf{\textbf{#1}}}

% set symbol
\usepackage[mathscr]{euscript}

% empty set
\let\emptyset\varnothing

% qed
\newcommand{\qeds}{\hfill\qedsymbol}

% math bold face
\newcommand{\bm}{\mathbf}

% argmax
\DeclareMathOperator*{\argmax}{argmax}
\DeclareMathOperator*{\argmin}{argmin}

% colorful reference
\usepackage{hyperref}
\usepackage{color}
\definecolor{darkred}{rgb}{0.7,0,0}
\definecolor{darkgreen}{rgb}{0,0.5,0}
\hypersetup{colorlinks=true,
        linkcolor=darkred,
        citecolor=darkgreen}
\urlstyle{same}

% independence symbol
\makeatletter
\newcommand*{\indep}{%
  \mathbin{%
    \mathpalette{\@indep}{}%
  }%
}
\newcommand*{\nindep}{%
  \mathbin{%                   % The final symbol is a binary math operator
    \mathpalette{\@indep}{\not}% \mathpalette helps for the adaptation
                               % of the symbol to the different math styles.
  }%
}
\newcommand*{\@indep}[2]{%
  \sbox0{$#1\perp\m@th$}%        box 0 contains \perp symbol
  \sbox2{$#1=$}%                 box 2 for the height of =
  \sbox4{$#1\vcenter{}$}%        box 4 for the height of the math axis
  \rlap{\copy0}%                 first \perp
  \dimen@=\dimexpr\ht2-\ht4-.2pt\relax
  \kern\dimen@
  {#2}
  \kern\dimen@
  \copy0 %                       second \perp
} 
\makeatother

%%%%%%%%%%%%%%%%%%%%%%%%%%%%%%%%%%%%%%%%%%%%%%%%%%%%%%%%%%%%%%%%%%%%%%%%
%%%%%%%%%%%%%%%%%%%%%%%%% Begin document here %%%%%%%%%%%%%%%%%%%%%%%%%%
%%%%%%%%%%%%%%%%%%%%%%%%%%%%%%%%%%%%%%%%%%%%%%%%%%%%%%%%%%%%%%%%%%%%%%%%
\begin{document}
\section*{Problem 6.3}
%
\textbf{Key routines in the program.}
%
\noindent
For Gibbs node-by-node sampler, the key routines are 
\begin{enumerate}
	\item Initialization of node values (e.g., all $+1$ or all $-1$). Initialization of edge potentials. (Figure~\ref{f:code}a)
	\item In each iteration on a node, compute the probability of choosing $+1$, based on edge potential
	  	  and the values of the neighbor nodes. The probability is computed
	  	  in Equation~(1) from Problem 6.2. (Figure~\ref{f:code}b)
	\item A loop through all nodes in each sampling iteration. (Figure~\ref{f:code}c)
\end{enumerate}
%
For Gibbs comb block sampler, the key routines are
\begin{enumerate}
	\item Initialization of node values (e.g., all $+1$ or all $-1$). Initialization of edge potentials. This
	      step is the same as Gibbs node-by-node sampler. (Figure~\ref{f:code}a)
	\item A message update routine to aggregate messages for specified destination (requiring source message all aggregated). 
	      (Figure~\ref{f:code}d)
	\item A serial belief propagation routine that calls the message update following a traversal order
	      (from the proper indexing in Problem 6.2(b)). (Figure~\ref{f:code}e)
	\item A scheme to determine the tree traversal order (indexing) for the two combs. (Figure~\ref{f:code}f)
	\item In each iteration on a node, update the node and edge potential in the tree path, and then
	      perform serial belief propagation to get all the messages on one direction.
	      Start sampling from the marginals. Based on the sampled value and passed messages,
	      sample the conditional probability using Equation~(2) from Problem 6.2(b). (Figure~\ref{f:code}g)
	\item A loop through all nodes in each sampling iteration. (Figure~\ref{f:code}h)
\end{enumerate}

\noindent
\textbf{Visualization of the samples.}
\noindent
We run the Gibbs node-by-node and the comb-shaped block sampler for 1000 iterations, initializing the nodes by
all $+1$ or all $-1$. The samples are visualized in Figure~\ref{f:63a}.
%
\begin{figure*}[h]
\captionsetup[subfigure]{labelformat=empty}
\centering
%
\begin{subfigure}[t]{0.09\textwidth}
\centering
\includegraphics[width=\textwidth]{./computational/results/gibbs_node_sampler_positive_iter_0.png}
\vspace{-0.6cm}
\end{subfigure}\hspace{0.001\textwidth}
%
%
\begin{subfigure}[t]{0.09\textwidth}
\centering
\includegraphics[width=\textwidth]{./computational/results/gibbs_node_sampler_positive_iter_100.png}
\vspace{-0.6cm}
\end{subfigure}\hspace{0.001\textwidth}
%
%
\begin{subfigure}[t]{0.09\textwidth}
\centering
\includegraphics[width=\textwidth]{./computational/results/gibbs_node_sampler_positive_iter_200.png}
\vspace{-0.6cm}
\end{subfigure}\hspace{0.001\textwidth}
%
%
\begin{subfigure}[t]{0.09\textwidth}
\centering
\includegraphics[width=\textwidth]{./computational/results/gibbs_node_sampler_positive_iter_300.png}
\vspace{-0.6cm}
\end{subfigure}\hspace{0.001\textwidth}
%
%
\begin{subfigure}[t]{0.09\textwidth}
\centering
\includegraphics[width=\textwidth]{./computational/results/gibbs_node_sampler_positive_iter_400.png}
\vspace{-0.6cm}
\end{subfigure}\hspace{0.001\textwidth}
%
%
\begin{subfigure}[t]{0.09\textwidth}
\centering
\includegraphics[width=\textwidth]{./computational/results/gibbs_node_sampler_positive_iter_500.png}
\vspace{-0.6cm}
\end{subfigure}\hspace{0.001\textwidth}
%
%
\begin{subfigure}[t]{0.09\textwidth}
\centering
\includegraphics[width=\textwidth]{./computational/results/gibbs_node_sampler_positive_iter_600.png}
\vspace{-0.6cm}
\end{subfigure}\hspace{0.001\textwidth}
%
%
\begin{subfigure}[t]{0.09\textwidth}
\centering
\includegraphics[width=\textwidth]{./computational/results/gibbs_node_sampler_positive_iter_700.png}
\vspace{-0.6cm}
\end{subfigure}\hspace{0.001\textwidth}
%
%
\begin{subfigure}[t]{0.09\textwidth}
\centering
\includegraphics[width=\textwidth]{./computational/results/gibbs_node_sampler_positive_iter_800.png}
\vspace{-0.6cm}
\end{subfigure}\hspace{0.001\textwidth}
%
%
\begin{subfigure}[t]{0.09\textwidth}
\centering
\includegraphics[width=\textwidth]{./computational/results/gibbs_node_sampler_positive_iter_900.png}
\vspace{-0.6cm}
\end{subfigure}\hspace{0.001\textwidth}
%
%
\begin{subfigure}[t]{0.09\textwidth}
\centering
\includegraphics[width=\textwidth]{./computational/results/gibbs_node_sampler_negative_iter_0.png}
\vspace{-0.6cm}
\end{subfigure}\hspace{0.001\textwidth}
%
%
\begin{subfigure}[t]{0.09\textwidth}
\centering
\includegraphics[width=\textwidth]{./computational/results/gibbs_node_sampler_negative_iter_100.png}
\vspace{-0.6cm}
\end{subfigure}\hspace{0.001\textwidth}
%
%
\begin{subfigure}[t]{0.09\textwidth}
\centering
\includegraphics[width=\textwidth]{./computational/results/gibbs_node_sampler_negative_iter_200.png}
\vspace{-0.6cm}
\end{subfigure}\hspace{0.001\textwidth}
%
%
\begin{subfigure}[t]{0.09\textwidth}
\centering
\includegraphics[width=\textwidth]{./computational/results/gibbs_node_sampler_negative_iter_300.png}
\vspace{-0.6cm}
\end{subfigure}\hspace{0.001\textwidth}
%
%
\begin{subfigure}[t]{0.09\textwidth}
\centering
\includegraphics[width=\textwidth]{./computational/results/gibbs_node_sampler_negative_iter_400.png}
\vspace{-0.6cm}
\end{subfigure}\hspace{0.001\textwidth}
%
%
\begin{subfigure}[t]{0.09\textwidth}
\centering
\includegraphics[width=\textwidth]{./computational/results/gibbs_node_sampler_negative_iter_500.png}
\vspace{-0.6cm}
\end{subfigure}\hspace{0.001\textwidth}
%
%
\begin{subfigure}[t]{0.09\textwidth}
\centering
\includegraphics[width=\textwidth]{./computational/results/gibbs_node_sampler_negative_iter_600.png}
\vspace{-0.6cm}
\end{subfigure}\hspace{0.001\textwidth}
%
%
\begin{subfigure}[t]{0.09\textwidth}
\centering
\includegraphics[width=\textwidth]{./computational/results/gibbs_node_sampler_negative_iter_700.png}
\vspace{-0.6cm}
\end{subfigure}\hspace{0.001\textwidth}
%
%
\begin{subfigure}[t]{0.09\textwidth}
\centering
\includegraphics[width=\textwidth]{./computational/results/gibbs_node_sampler_negative_iter_800.png}
\vspace{-0.6cm}
\end{subfigure}\hspace{0.001\textwidth}
%
%
\begin{subfigure}[t]{0.09\textwidth}
\centering
\includegraphics[width=\textwidth]{./computational/results/gibbs_node_sampler_negative_iter_900.png}
\vspace{-0.6cm}
\end{subfigure}\hspace{0.001\textwidth}
%
%
\begin{subfigure}[t]{0.09\textwidth}
\centering
\includegraphics[width=\textwidth]{./computational/results/gibbs_comb_sampler_positive_iter_0.png}
\vspace{-0.6cm}
\end{subfigure}\hspace{0.001\textwidth}
%
%
\begin{subfigure}[t]{0.09\textwidth}
\centering
\includegraphics[width=\textwidth]{./computational/results/gibbs_comb_sampler_positive_iter_100.png}
\vspace{-0.6cm}
\end{subfigure}\hspace{0.001\textwidth}
%
%
\begin{subfigure}[t]{0.09\textwidth}
\centering
\includegraphics[width=\textwidth]{./computational/results/gibbs_comb_sampler_positive_iter_200.png}
\vspace{-0.6cm}
\end{subfigure}\hspace{0.001\textwidth}
%
%
\begin{subfigure}[t]{0.09\textwidth}
\centering
\includegraphics[width=\textwidth]{./computational/results/gibbs_comb_sampler_positive_iter_300.png}
\vspace{-0.6cm}
\end{subfigure}\hspace{0.001\textwidth}
%
%
\begin{subfigure}[t]{0.09\textwidth}
\centering
\includegraphics[width=\textwidth]{./computational/results/gibbs_comb_sampler_positive_iter_400.png}
\vspace{-0.6cm}
\end{subfigure}\hspace{0.001\textwidth}
%
%
\begin{subfigure}[t]{0.09\textwidth}
\centering
\includegraphics[width=\textwidth]{./computational/results/gibbs_comb_sampler_positive_iter_500.png}
\vspace{-0.6cm}
\end{subfigure}\hspace{0.001\textwidth}
%
%
\begin{subfigure}[t]{0.09\textwidth}
\centering
\includegraphics[width=\textwidth]{./computational/results/gibbs_comb_sampler_positive_iter_600.png}
\vspace{-0.6cm}
\end{subfigure}\hspace{0.001\textwidth}
%
%
\begin{subfigure}[t]{0.09\textwidth}
\centering
\includegraphics[width=\textwidth]{./computational/results/gibbs_comb_sampler_positive_iter_700.png}
\vspace{-0.6cm}
\end{subfigure}\hspace{0.001\textwidth}
%
%
\begin{subfigure}[t]{0.09\textwidth}
\centering
\includegraphics[width=\textwidth]{./computational/results/gibbs_comb_sampler_positive_iter_800.png}
\vspace{-0.6cm}
\end{subfigure}\hspace{0.001\textwidth}
%
%
\begin{subfigure}[t]{0.09\textwidth}
\centering
\includegraphics[width=\textwidth]{./computational/results/gibbs_comb_sampler_positive_iter_900.png}
\vspace{-0.6cm}
\end{subfigure}\hspace{0.001\textwidth}
%
%
\begin{subfigure}[t]{0.09\textwidth}
\centering
\includegraphics[width=\textwidth]{./computational/results/gibbs_comb_sampler_negative_iter_0.png}
\vspace{-0.6cm}
\end{subfigure}\hspace{0.001\textwidth}
%
%
\begin{subfigure}[t]{0.09\textwidth}
\centering
\includegraphics[width=\textwidth]{./computational/results/gibbs_comb_sampler_negative_iter_100.png}
\vspace{-0.6cm}
\end{subfigure}\hspace{0.001\textwidth}
%
%
\begin{subfigure}[t]{0.09\textwidth}
\centering
\includegraphics[width=\textwidth]{./computational/results/gibbs_comb_sampler_negative_iter_200.png}
\vspace{-0.6cm}
\end{subfigure}\hspace{0.001\textwidth}
%
%
\begin{subfigure}[t]{0.09\textwidth}
\centering
\includegraphics[width=\textwidth]{./computational/results/gibbs_comb_sampler_negative_iter_300.png}
\vspace{-0.6cm}
\end{subfigure}\hspace{0.001\textwidth}
%
%
\begin{subfigure}[t]{0.09\textwidth}
\centering
\includegraphics[width=\textwidth]{./computational/results/gibbs_comb_sampler_negative_iter_400.png}
\vspace{-0.6cm}
\end{subfigure}\hspace{0.001\textwidth}
%
%
\begin{subfigure}[t]{0.09\textwidth}
\centering
\includegraphics[width=\textwidth]{./computational/results/gibbs_comb_sampler_negative_iter_500.png}
\vspace{-0.6cm}
\end{subfigure}\hspace{0.001\textwidth}
%
%
\begin{subfigure}[t]{0.09\textwidth}
\centering
\includegraphics[width=\textwidth]{./computational/results/gibbs_comb_sampler_negative_iter_600.png}
\vspace{-0.6cm}
\end{subfigure}\hspace{0.001\textwidth}
%
%
\begin{subfigure}[t]{0.09\textwidth}
\centering
\includegraphics[width=\textwidth]{./computational/results/gibbs_comb_sampler_negative_iter_700.png}
\vspace{-0.6cm}
\end{subfigure}\hspace{0.001\textwidth}
%
%
\begin{subfigure}[t]{0.09\textwidth}
\centering
\includegraphics[width=\textwidth]{./computational/results/gibbs_comb_sampler_negative_iter_800.png}
\vspace{-0.6cm}
\end{subfigure}\hspace{0.001\textwidth}
%
%
\begin{subfigure}[t]{0.09\textwidth}
\centering
\includegraphics[width=\textwidth]{./computational/results/gibbs_comb_sampler_negative_iter_900.png}
\vspace{-0.6cm}
\end{subfigure}\hspace{0.001\textwidth}
%
\caption{Visualization of the Gibbs samples.
         1st row: node-by-node samples, initialized with all $+1$.
         2nd row: node-by-node samples, initialized with all $-1$.
         3rd row: block samples, initialized with all $+1$.
         4th row: block samples, initialized with all $-1$.
         Each column corresponds to $\{1, 100, 200, \cdots, 900\}$ iterations.
         The $+1$ region is denoted as white, $-1$ as black.}
\label{f:63a}
\end{figure*}
%

\noindent
\textbf{Analysis.}
\noindent
By symmetry, there should not be a bias between $+1$ or $-1$ samples.
%
Thus, over a long time, the samples should give $+1$ or $-1$ with the same probability for all nodes.
%
However, since $\theta=0.45$ is relatively large, the values of the nodes tend to stick with each
other; and the mixing time can be long.
%
We plot the average value of all nodes over $20,000$ Markov process steps for both sampler in Figure~\ref{f:63}.
%
From the figure, Gibbs block sampler oscillates between the $+1$ and $-1$ regions more often;
and the mixing time should be empirically shorter. 
%

In terms of per-sweep running time, the complexity is both $O(n)$. Since the block sampling scheme has
an extra step for message passing and updating the node potentials from the observation of the other block,
the empirical runtime for the block sampler is slightly longer ($\sim 2\times$ longer in my implementation).

Also from figure~\ref{f:63}, different initializations affect which $+1/-1$ region to sample first the
but don't empirically affect the mixing time.
\\

\begin{figure*}[h]
\centering
\vspace{-0.6cm}
\includegraphics[width=\textwidth]{./63.pdf}
\vspace{-0.8cm}
\caption{Mixing behavior of Gibbs samplers. Left: node-by-node sampler. Right: comb-shaped block sampler.}
\label{f:63}
\end{figure*}

\noindent
\textbf{Code snippets.}
The code snippets are included in Figure~\ref{f:code}.
%
\begin{figure*}[h]
\centering
\vspace{-0.2cm}
\includegraphics[width=0.98\textwidth]{./computational/code_screenshots/code.pdf}
\vspace{-0.2cm}
\caption{Code snippets for Gibbs sampler.}
\label{f:code}
\end{figure*}

\pagebreak

\end{document}